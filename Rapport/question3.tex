Soient $A$ le 1-arbre optimal et $ T $ la tournée optimale.
\begin{itemize}
	\item Si tous les sommets de $ A $ sont de degré $ 2 $, alors le 1-arbre est une tournée. Or $ A $ est optimale donc $ A \equiv T $.
	\item Sinon il existe des sommets de $ A $ de degrés autres que $ 2 $, soient les sommets $ s_{1}, s_{2}, \dots, s_{k} $ avec $ k \in \mathbb{N} \\ k \leq n $ tels que $ \forall i \leq k, d(s_{i}) \not = 2 $.
	\begin{itemize}
		\item Si $ d(s_{i}) = 1 $, alors cette feuille manque d'un voisin pour faire parti d'un tour.
		\item Si $ d(s_{i}) > 2$, alors ce sommet possêde "trop" de voisins pour faire parti d'un tour. 
	\end{itemize}
	Or les arêtes d'un 1-arbre optimale, sont toutes optimales au sens de leur poid. Donc l'arête ajoutée aux feuilles de $ A $ pour faire de $ A $ une tournée ne sont pas de poid optimale. Soit $ (a_{i})_{i \in \mathbb{N}}  $ le poid des arêtes ajoutées. De plus, les arêtes retirées aux noeuds de degrés supérieur à $ 2 $ sont optimales. Soit $ (r_{i})_{i \in \mathbb{N}} $ le poid de chaque arête retirée. Alors on a:
	\begin{align*}
		d(T) &= d(A) + \sum_{i} a_{i} - \sum_{i} r_{i}\\
		\intertext{Or les $ r_{i} $ sont de poid optimale contrairement aux $ a_{i} $. Donc,}
		\sum_{i} r_{i} &\leq \sum_{i} a_{i}
		\intertext{Donc,}
		d(T) \geq d(A)
	\end{align*}
\end{itemize}

Donc un 1-arbre optimale est de poid inférieure ou égale à la tournée optimale. Donc un 1-arbre optimale est une borne inférieur de la longueur d'une tournée optimale.