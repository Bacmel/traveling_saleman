%%%%%%%%%%%%%%%%%%%%%%%%%%%%%%%%%%%%%%%%%%%%%%%%%%%%%%%%%%%%%%%%%%%%%
%
% Original Source: http://www.howtotex.com
% Date: February 2014
% 
% This is a title page template which be used for articles & reports.
% 
% This is the modified version of the original Latex template from
% aforementioned website.
% 
%%%%%%%%%%%%%%%%%%%%%%%%%%%%%%%%%%%%%%%%%%%%%%%%%%%%%%%%%%%%%%%%%%%%%%

\documentclass[12pt]{report}
\usepackage[a4paper]{geometry}
\usepackage[myheadings]{fullpage}
\usepackage{fancyhdr}
\usepackage{lastpage}
\usepackage{amsmath}
\usepackage{amsfonts}
\usepackage{amssymb}
\usepackage{graphicx, wrapfig, subcaption, setspace, booktabs}
\usepackage[T1]{fontenc}
\usepackage[utf8]{inputenc}
\usepackage[font=normalsize, labelfont=bf]{caption}
\usepackage{lmodern}
%\usepackage{fourier}
\usepackage[protrusion=true, expansion=true]{microtype}
\usepackage[french]{babel}
\usepackage{sectsty}
\usepackage{url, lipsum}

\newcommand{\HRule}[1]{\rule{\linewidth}{#1}}
\onehalfspacing
\setcounter{tocdepth}{5}
\setcounter{secnumdepth}{5}

%-------------------------------------------------------------------------------
% HEADER & FOOTER
%-------------------------------------------------------------------------------
\pagestyle{fancy}
\fancyhf{}
\setlength\headheight{15pt}
\fancyhead[L]{Projet Algorithmique - Traveling Saleman Problem}
\fancyhead[R]{\includegraphics[width = 20mm]{LSU.png}
			\includegraphics[width = 29mm]{PSH.png}}
\fancyfoot[R]{\thepage\ / \pageref{LastPage}}
%-------------------------------------------------------------------------------
% TITLE PAGE
%-------------------------------------------------------------------------------

\begin{document}

\title{
	    \normalsize \textsc{ROB3 S6 - Algorithmique}
		\\ [1.0cm]
		\HRule{0.5pt} \\
		\LARGE \textbf{\uppercase{Problème du voyageur\\du commerce}}
		\HRule{2pt} \\ [0.5cm]
		\normalsize \today{} \vspace*{5\baselineskip}}
\date{}
\author{
		Florian CORMÉE \\
		Hugo DUARTE}
\maketitle
\tableofcontents
%\sectionfont{\scshape}
\newpage

\section*{Introduction}
\addcontentsline{toc}{section}{Introduction}

Du contenue

\section*{Partie n} % Dessin main levé+choix design selon contrainte
\addcontentsline{toc}{section}{Partie n}

\subsection*{Question 1}
\addcontentsline{toc}{subsection}{Question 1}
L'algorithme de prim consiste à sélectionner l'arête de poid minimum tant qu'il y a des sommets qui ne font pas partie du graphe. En utilisant, une liste de voisin pour représenter le graphe, et un tas pour lister les voisins, à chaque itération, on récupère le tête du tas qui se réalise en $O(\log(n))$, puis on met à jour les poids des voisins du sommet sélectionné qui demande une boucle sur les voisins et le repositionnement du sommet dans le tas cette dernière opération est réalisée en $O(\log(n))$. Donc pour un tour de boucle l'opération est en $O((1 + deg(s))\times\log(n))$. Il y a $n$ tour de boucle pour que l'arbre soit couvrant. Donc l'algorithme de Prim a une complexité de $O((n+m) \log(n))$ pour un graphe sous forme de liste de voisins et en utilisant un tas.

\subsection*{Question 3}
\addcontentsline{toc}{subsection}{Question 2}
Soient $A$ le 1-arbre optimal et $ T $ la tournée optimale.
\begin{itemize}
	\item Si tous les sommets de $ A $ sont de degré $ 2 $, alors le 1-arbre est une tournée. Or $ A $ est optimale donc $ A \equiv T $.
	\item Sinon il existe des sommets de $ A $ de degrés autres que $ 2 $, soient les sommets $ s_{1}, s_{2}, \dots, s_{k} $ avec $ k \in \mathbb{N} \\ k \leq n $ tels que $ \forall i \leq k, d(s_{i}) \not = 2 $.
	\begin{itemize}
		\item Si $ d(s_{i}) = 1 $, alors cette feuille manque d'un voisin pour faire parti d'un tour.
		\item Si $ d(s_{i}) > 2$, alors ce sommet possêde "trop" de voisins pour faire parti d'un tour. 
	\end{itemize}
	Or les arêtes d'un 1-arbre optimale, sont toutes optimales au sens de leur poid. Donc l'arête ajoutée aux feuilles de $ A $ pour faire de $ A $ une tournée ne sont pas de poid optimale. Soit $ (a_{i})_{i \in \mathbb{N}}  $ le poid des arêtes ajoutées. De plus, les arêtes retirées aux noeuds de degrés supérieur à $ 2 $ sont optimales. Soit $ (r_{i})_{i \in \mathbb{N}} $ le poid de chaque arête retirée. Alors on a:
	\begin{align*}
		d(T) &= d(A) + \sum_{i} a_{i} - \sum_{i} r_{i}\\
		\intertext{Or les $ r_{i} $ sont de poid optimale contrairement aux $ a_{i} $. Donc,}
		\sum_{i} r_{i} &\leq \sum_{i} a_{i}
		\intertext{Donc,}
		d(T) \geq d(A)
	\end{align*}
\end{itemize}

Donc un 1-arbre optimale est de poid inférieure ou égale à la tournée optimale. Donc un 1-arbre optimale est une borne inférieur de la longueur d'une tournée optimale.


\section*{Conclusion}
\addcontentsline{toc}{section}{Conclusion}

Du contenue

\end{document}
